\documentclass[12pt, notitlepage]{report}

\usepackage{html}

% For convenience when quoting code
\newcommand{\code}[1]{\textsf{#1}}
\newcommand{\codeblock}[1]{\indent \textsf{#1}}

\title{Peep Server Documentation}
\author{Michael Gilfix
(\htmladdnormallink{mgilfix@eecs.tufts.edu}
                   {mailto:mgilfix@eecs.tufts.edu})
}

\begin{document}

\maketitle

This documentation is current as of Peep: The Network Auralizer version
0.5.0.

This documentation is the general documentation for Peep and includes a
description of how to setup the Peep server, how to configure Peep within a
nework, as well as Peep's internals. In other words, this document is for
everyone.

If you're looking for a description of specific Peep clients, or a
description of the client API, these are mantained in a separate
document that can be found
\htmladdnormallinkfoot{here}{http://peep.sourceforge.net/docs/peep-client.html}.

\tableofcontents

\bigskip

\chapter{Introduction}

This section provides a gentle introduction to Peep.

\section{What is Peep?}

This is the Peep server documentation. Peep is a network monitoring tool
that represents network information via audio output. Network diagnosis
with Peep is made not only based on singular network events, but on whether
the network as a whole ''sounds normal''. This document gives you information
about how to install Peep on your system and use Peep effectively. The
document also provides developers with information about Peep's internals
and tells potential developers how they can get involved in the project.

At this point, many people might think to themselves, ''Well, won't
I get annoyed? I don't like computer sounds...''. A lot of effort in
the development of Peep went to providing a solution to this very
problem. Peep doesn't beep at you and it certainly doesn't play a sequence
of beeps. Rather, Peep uses sounds from natural environments. Sounds are
mapped to arbitrary network events and are mixed together to create an
overall sound ambiance. Because the sounds used are taken from natural
environments, the resulting ambiance is pleasing to listen to (and can
be tweaked to your preferences).  Peep is also meant to be played in the
background. Because humans are very adept at noticing changes quickly and
subconsciously, network monitoring with Peep won't interfere with other
work.

Peep was originally developed by Michael Gilfix at
\emph{Tufts University}. It now resides in its new home on
\htmladdnormallinkfoot{Sourceforge}
{http://www.sourceforge.net/projects/peep}

\section{Revision Information}

This documentation is available in html format and its original latex source.

\begin{itemize}
\item Version 3.0 - Changed format to latex (this will be the last format
      change), and updated to 0.5.0.
\item Version 2.2 - Fixed misc spelling and grammar mistakes at the time of
      the 0.4.5 release.
\item Version 2.1 - April 1st, 2001 - The 2.0 version has now been converted
      to SGML, and can now be converted to other formats thanks to Xavier
      Renault.
\item Version 2.0 - Mar 25th, 2001 - Second version, modified extensively for
      the release of Peep 0.4.0. Available in html only.
\item Version 1.0 - Dec 23rd, 2000 - First version, created shortly after the
      release of 0.3.6. Available in html only.
\end{itemize}

If you find any spelling mistakes or want to add anything, just send patches
my way. In English preferrably, although I could do French.

\section{Other Sources of Information}

The best source of information is the sourceforce website. It contains
links to documentation, Peep demos, and other generally useful stuff.
Otherwise, the Peep mailing lists can also be helpful. There are currently
two existing mailing lists: \code{peep-general} and \code{peep-develop}. The
first is used to address general issues with Peep, and is at the time of
writing very low-traffic. The developer list is used for discussing new
ideas, coordinating between developers, as well as helping people debug
installations.

To subscribe to either mailing lists, please visit:
\htmladdnormallink{http://sourceforge.net/mail/?group\_id=15807}{http://sour
ceforge.net/mail/?group\_id=15807}. The Peep website can be accessed at:
\htmladdnormallink{http://peep.sourceforge.net}{http://peep.sourceforge.net}
. It contains a demo of Peep's capabilities (the demo used at the LISA 2000
conference) and the original \emph{best paper} presented at that conference
in the introduction page.

If you're interested in creating sounds for Peep or are
interested in the software used to do much of the sound
processing for Peep, please visit the Goldwave webpage at
\htmladdnormallink{http://www.goldwave.com}{http://www.goldwave.com}.

Also, if you're an ALSA user and have any questions
concerning your ALSA drivers, please visit their website at
\htmladdnormallink{http://www.alsa-project.org}{http://www.alsa-project.org}
.

\section{Feedback}

I rely on you, the reader, to make this document useful and
informative. If you have any suggestions, please send them to
\htmladdnormallink{mgilfix@eecs.tufts.edu}{mailto:mgilfix@eecs.tufts.edu}
and I'll try to incorporate those changes in the next version.

This document took me a while to write so even a ''thank you for'', ''I
appreciate this most'', or a ''this was not immediately clear to me'' email
would be appreciated.

\chapter{Getting Started With Peep}

This section will tell you what you need to get started with Peep and how to
get on your way.

\section{What You Need}

\begin{itemize}
\item A functioning unix system with development packages installed (gcc, make,
      etc\ldots).

\item A sound card and the privileges to use it.

\item Perl (For the clients) with access to CPAN (Several Perl libraries are
      required from CPAN). See the client documentation for more information.

\item The Peep source and a Peep sound package. Both of those can be obtained
      from the sourceforge website or via ftp at:
      \htmladdnormallink{ftp://download.sourceforge.net/pub/sourceforge/peep}
      {ftp://download.sourceforge.net/pub/sourceforge/peep}

\item The expat library for XML parsing:
      \htmladdnormallink{http://expat.sourceforge.net}
      {http://expat.sourceforge.net}.
\end{itemize}

I also suggest consulting the \code{INSTALL} text that is provided with each
distribution. It provides similar information to what's found here and might
be more current.

\section{Extracting}

This section assumes that you have obtained the source in \code{tar.gz}
format. If you obtained the code base as an rpm, I suggest doing a \code{man rpm}
to view the installation options. To unpack the archive in \code{/usr/local/src},
you'd most likely want to do:

\begin{quote}
\begin{verbatim}
cd /usr/local/src
tar -zxf ~/Peep-0.4.0.src.tar.gz
cd Peep-0.4.0; tar -zxf ~/Peep-Sounds.0.4.0.tar.gz
\end{verbatim}
\end{quote}

Note that the Peep sound package unpackges into \code{sounds/} and should be
unpacked into the source directory. Sounds will later be installed in the
correct location when a \code{make install} is executed.

\section{Compiling the Peep Server}

To compile the Peep server, you'll probably want to do a:

\begin{quote}
\begin{verbatim}
./configure
make
make install
\end{verbatim}
\end{quote}

To install Peep is a non-standard location, the \code{--prefix=$path$}
option must be supplied to \code{configure}.

Peep can be compiled to use different protocols to fit different needs.
As of 0.5.0, Peep supports three different protocols, which are enabled
using the following configure switches:

\noindent \code{--enable-tcp-server}: Compile the server to use the TCP
protocol for messaging and use UDP for broadcasting. (Default)

\noindent \code{--enable-udp-server}: Compile the server to use UDP for all
communications.

\noindent \code{--enable-openssl}: Compile the server to use SSL for
messaging and UDP for broadcasting.

The clients can then be configured to use the appropriate protocol at
runtime. See the client documentation for further details.

\code{configure} also understands some other useful options:

\noindent \code{--enable-debug=(0-5)}: This sets the level of debugging
output. The default is to use the lowest level of debugging, 0.  To see a
list of what information will be printed out with each level of debugging,
type \code{./configure --help}.

\noindent \code{--with-dynamic-volume}: This is the default setting and
enables dynamic calculations of volume levels in the mixer.

\noindent \code{--with-static-volume}: This disables dynamic volume and
uses a fixed volume for each sound. This can fix any sort of hesitations
in that might result from sound clipping. But this shouldn't happen,
so you shouldn't have to do this =) Static volume might give a minor
performance increase if your server is getting hit with a lot of events
quickly. However, static volume greatly reduces sound output level, so get
ready to crank up the stereo. (Default)

\noindent \code{--with-alsa-driver}: This compiles \code{peepd} using the
ALSA libraries, rather than \code{/dev/audio}. This code currently support
ALSA 0.9.x. Most linux systems ship with this version as of time of writing.

\noindent \code{--with-oss-driver}: This tells \code{peepd} to use standard
\code{/dev/audio} support. This is the default and so this option will most
likely never be used.

\noindent \code{--with-expat=$path$}: This specifies an alternate location
for the expat library and include prefix.

\noindent \code{--with-openssl-includes}: This specifies the directory
containing the openssl include files if they cannot be found by
\code{configure}.

\noindent \code{--with-openssl-libraries}: This specifies the directory
containing the openssl library files if they cannot be found by
\code{configure}.

After \code{peepd} has been compiled, the next step is to configure
\code{peep.conf} and deploy the clients. Deploying and configuring the
client utlities is a little more involved and is described in their
respective sections.

\section{Running the Peep Server}

\code{peepd} takes several arguments on the command line. This section
documents some of the more interesting features. To see the complete
list, type \code{\$ ./peepd --help} at the command prompt. Once started,
\code{peepd} will log all its output to standard error by default.  So, if
you're starting the server in daemon mode (the default) and want to suppress
the output, I suggest:

\codeblock{peepd $>$ /dev/null 2$>$\&1}

If you've compiled \code{peepd} in debugging mode or want to make sure that
everything is running smoothly, I suggest redirecting the logs:

\codeblock{peepd -l /var/local/peepd/peepd.log}

Also, the \code{--nodaemon} option can be a handy way to watch what's going
on and make troobleshooting easier.

When running in daemon mode \code{peepd} logs its pid to
\code{/var/run/peepd.pid} by default. Depending on which user you use to run
the server or for whatever reason if you do not have access to write out
a file in \code{/var/run}, you can tell peep to write its pid to another
location by supplying the --pidfile option:

\codeblock{peepd --pidfile=/other/location/peepd.pid}

You can also tell the server how many voices (simultaneous channels) to use
when performing real-time mixing. This allows you to toggle performance
versus the needs of your network. A greater number of voices causes
\code{peepd} to chew up more CPU time but introduces less delay in a busy
network.  The default is 16 voices but you can change this on the command
line with the -v option:

\codeblock{peepd -v32}

\code{peepd} also supports a feature for recording and playing back events,
which is described in the next section

\section{Peep's Playback and Recording Features}

This section describes a feature in \code{peepd} whereby the server can
record incoming events for later playback and a closer listen. The events
are stored in ''round-robin'' fashion in a playback file specified on
the command-line. The maximum number of events is set as a define in the
\code{playback.h} file in the server code and defaults to 3200 events. The
size of the plaback file varies greatly with the amount of XML data that
might be sent through the messaging structure but 3200 events should
generate a playback file well under 1 MByte. There are five command-line
options used to get the playback code running:

\begin{quote}
--playback-mode
--record-mode
--record-file=$file$
--start-time=$data$
--end-time=$data$
\end{quote}

The recording file defaults to \code{/var/log/peeplog}, which is set via
a \#define in main.h. Otherwise, events are recorded to the file specified
by the \code{--record-file} option. \code{peepd} does not do recording
by default. To kick \code{peepd} into recording mode, you need to use
the \code{--record-mode} option and optionally the \code{--record-file}
option. An example:

\codeblock{./peepd --record-mode -record-file=test.log}

Then to playback, you need to specify \code{--playback-mode} and
\code{peepd} will playback the events in the log and then exit. As always,
\code{--record-file} is optional. An example:

\codeblock{./peepd --playback-mode --record-file=test.log}

Since there are many events in a single log file, you'll probably
want to play back sounds between a specific time. \code{peepd} lets
you do this through use of the \code{--start-time=$start time$} and
\code{--end-time=$end time$} options. If a \code{--start-time} is given
without an \code{--end-time}, \code{peepd} will play back from the start
time to the last event in the file. The date format used is the same as the
output of date except without the specifier for EST/PST. The date format
used is defined in \code{playback.h} as a string to \code{strptime}. An
example:

\codeblock{./peepd --playback-mode -f testlog -t 'Wed Mar 14 22:15:45 2001'
-e 'Wed Mar 14 22:17:00 2001'}

Note that the 24 hr time clock is used. It's less ambiguous and simpler.

\chapter{Overview of How Peep Works}

The purpose of this section is to give you a brief idea of how Peep does
what it does and how to setup your Peep system effectively.

\section{Brief Summary of it All}

The Peep architecture is a producer/consumer architecture where the clients
are producers and the servers consumers. Clients are meant to be distributed
around the network, scanning for information at the source. Upon discovery
of an event, the client sends off the appropriate information to its
respective server(s) and the server(s) represent that data accordingly. A
simple example scenario would be a single \code{peepd} server running on
a centralized server and a monitoring system that monitors logs on three
different machines. All three monitoring clients report their events to the
server.  Events can be distinguished by the sounds they play or even stereo
location. The server has no concept of which client is which. It simply
represents the raw network event data as it receives it. This scenario can
easily be expanded to include multiple servers or many more clients, each
scanning for different types of information on each machine. Because the
sounds used come from natural environments, sound configurations are not
limited by musical or harmonic combinations and are as ''scalable'' as your
what your ear can handle.

All servers and clients are configured using the same configuration file,
\code{peep.conf}. The configuration file includes sections containing
information specific to each client as well as the ''class'' or group
of servers that it belongs to. It also contains a list of events and
state sounds (See section 3.2). Mangement of clients and servers is made
easier through autodiscovery and leasing. Clients are meant to be run
as daemons on the respective hosts they monitor and bind with servers
automatically. Because all configuration information is stored in a single
configuration file, the file can be pushed around to each monitored host on
the network and changes made centrally.

Sounds provided with Peep are organized into different themes and
categories. Each theme contains four categories: events, coupled
events, state sounds, and heartbeat sounds. The current most complete
theme is a ''wetlands'' venue, and we're trying to organize several
sound themes to offer a wide variety of sounds to better meet diverse
tastes. The goal of Peep is to provide you with continual network status
information while remaining pleasing to the ear.  The Peep server is
very flexible in how it represents sounds, so I suggest playing around
with the sounds until you find a configuration that's comfortable for
you. And when you do, please share that configuration file with others
to ease their setup process. As of the 0.5.x series of Peep, a new XML
theme format exists to allow for easy interchanging of system setups
among various networks. Now, Peep can easily be configured to match
another customized enviroment by simply loading the theme file.  So
please send your system themes and configurations my way via email at:
\htmladdnormallink{mgilfix@eecs.tufts.edu}{mailto:mgilfix@eecs.tufts.edu}.

Network monitoring with Peep is based on the idea that humans find it very
easy to discern changes from the norm, discern what sounds right, and to
discern singular important sounds from a collection of many sounds. This
concept of ''normalcy'' allows you to diagnosis your network based on the
feel of the sound ambiance. In other words, things are normal when ''Peep
sounds like it did yesterday''.

\section{Events, States, and Heartbeats}

This section is very important for taking advantage of Peep's
representational capabilities as well as creating a pleasing sound
environment for network monitoring. If you're skimming through the doc,
I suggest taking the time to read this section with a little more care.

Sound representation in Peep is divided into three different
categories: Events in networks are things that occur once, naturally
represented by a a single peep or chirp. Network states represent ongoing
network measurements by changing the type, volume, or stereo position of an
ongoing background sound. Heartbeats represent the existence or frequency
of occurrence of an ongoing network state by playing a sound at varying
intervals, such as by changing the frequency of cricket chrips.

Peep represents discrete events by playing a single natural sound every time
the event occurs, such as a bird chirp or a woodpecker's peck. The sounds
used are short and staccato in nature and easily distinguishable by the
listener. Also, certain network events tend to happen in couples or with a
strong correlation to each other. For these sounds, you may wish to select
complementary sounds, so as to better represent coupling.  An example of
this is the bird sounds chosen in the Peep demo. They match incoming and�
outgoing email, which we noticed was often received/sent several times in
single smtp session between mail servers. Another thing to note is that
several different sounds can be associated with a single event. When the
event occurs, the server picks one randomly. Thus, if you have sounds that
sounds very similar, I suggest mapping them to the same event to give more
variation and make the sound ambiance more natural.

State sounds correspond to measurements or weights describing the magnitude
of something, such as the load average or the number of users on a given
machine. Unlike events, which are only played when Peep is notified of them,
Peep plays state information constantly and need only be signaled when state
sounds should change. Peep represents a state with a continuous stream of
background sounds, like a waterfall or wind. Each state is internally scaled
to vary from extremely quiet to loud and obnoxious. Background sounds should
be soothing while the network is functioning normally and annoying when
action is necessary.

Heartbeats are sounds that occur at constant intervals, analogous to
crickets chirping at night. A common folk tale is that one can tell the
temperature from the frequency of cricket chirps; likewise we can represent
network load as a similar function. Intermittent chirps might mean low load,
while a chorus might mean high load. Heartbeats can also report results of
an intermittent check (or ping) to see if a given machine, device or server
is functioning properly. Note that the server does not distinguish between
heartbeat events and regular events.  Rather, the event is sent to the
server at the appropriate interval, as chosen by the monitoring client.

\section{Auto-Discovery and Leasing}

The auto-discovery and leasing mechanism is there to make your life easier.
While the role of the mechanism changes slightly depending on the underlying
protocol used in your Peep server, it's general role is to allow servers
and clients to automatically find each other via broadcasting and only send
information while the server is functioning correctly, i.e, the connection
is valid. In the case of a pure UDP Peep system, the connectionless nature
makes it difficult for clients to determine whether a server is still
listening and functioning correctly. A connection leasing mechanism is then
used to circumvent this problem.

Peep's autodiscovery mechanism uses a domain-class concept to maintain
bindings between clients and their respective servers. When a server
initializes, it broadcasts its existence to the subnets associated with
its classes and announces the classes of which it is a part. The clients
that are members of those classes register themselves with the server and
begin sending it data. Conversely, should a client start up and broadcast
its existence, the server associated with its class will tell it to begin
sending. \emph{A broadcast only occurs once during the initialization of
each client or server}, after which a list of hosts is maintained on both
sides and communications are direct. Both clients and servers can belong
to multiple classes at the same time and clients can communicate with many
servers concurrently.

When running purely over UDP, leasing is used to ensure that clients do not
waste network bandwidth and system resources sending packets to servers that
are no longer listening.  The server sends a lease time to the client during
auto-discovery. Just before the lease expires, the server tells the client
to renew the lease.  The client responds by telling the server that it is
still alive and still needs to know about lease information. If the client
has not heard from a server after the least time has expired, it will no
longer send packets to that server. Similarly if a server does not receive
lease acknowledgement from a client, it will no longer attempt to renew its
lease with that client.

Using auto-discovery and leasing is entirely optional. Clients can be run
alternatively as ''dumb'' clients, as Auto-discovery and leasing isn't the
solution in every situation. Clients can have their auto-discovery disabled
with the \code{--noautodiscovery} option and then use the \code{--server}
and \code{-port} options to figure out where to send their data.

\section{Testing and Experimenting with Sounds}

\textit{... While \code{keytest} used to be the best way to experiment with
server sounds, this will change with the 0.5.x series. So consider this
section under construction.}

\section{Sounds, Formats, etc...}

Sounds in Peep are stored in raw PCM signed-16 bit CD quality (44100 kHz,
stereo) format. Oh yeah, and they're little endian too $=)$ If that sounds
scary, fear not. If you're on a machine whose byte order is like Intel's,
just load up your CD quality waves (They should be at 44100 Khz) and convert
them to .snd format. Then just rename the sound files according the servers
convention and you're all set.

To play a sound under GNU/Linux, you can do:

\codeblock{\$ bplay -s 44100 -b 16 -S $file$}

\code{bplay} can be retrieved from its homepage at:
\htmladdnormallink{http:://www.amberdata.demon.co.uk/bplay}
{http:://www.amberdata.demon.co.uk/bplay}.

Alternatively, if your distribution comes with the GNU/Linux "play" utility,
you can play a sound file with:

\codeblock{\$ play -c 2 -t raw -r 44100 --size=w -f s $file$}

Unfortunately, there is no direct support for .wav's at the moment but
hopefully that will be added in the future. In the meantime, many tools can
convert to .snd format and so you need not worry. Support for .mp3s in the
future is iffy. Although it would greatly compress the sound packages, it
also brings a large computational overhead (decoding). I want \code{peepd}
to be able to run on older machines without performance issues. However,
.mp3 might become an alternative in the future. If anyone is interested
in adding support for different sound formats, please email me:
\htmladdnormallink{mgilfix@eecs.tufts.edu}{mailto:mgilfix@eecs.tufts.edu}.

\chapter{Configuring peep.conf}

This section descibes how to configure the Peep server and its utilities. If
you're just setting up Peep and are having problems, this is probably the
place to look. \emph{Note that comments in \code{peep.conf}
are preceded by '\#' marks}.

\section{The ''General'' Section}

The ''general'' section describes some general attributes of the Peep
system. These attributes include things such as the version number of the
configuration file, and the path to the sound repository. These options are
parsed both by the servers and the clients.

A general section looks as follow:

\begin{quote}
\begin{verbatim}
general
    version <VERSION NUMBER>
    sound-path <PATH TO REPOSITORY>
end general
\end{verbatim}
\end{quote}

The version provides a means of indicating that the configuration file is
valid only for version of Peep that are greater than the current version
specified. This is useful for avoiding sharing newer configuration files
with older versions of Peep. As of 0.5.x, \code{peepd} will reject the
configuration file on start up if the version specified is too high. Version
numbers follow the same versioning scheme as peep: \code{x.x.x}.

The \code{sound-path} directive tells the Peep system where the top-level
directory of the sound respository is located. All subsequent refers to
sound paths will be relative to this directory. This greatly cuts down on
the typing needed to produce a configuration file, as well as makes the
configuration file much more straight-forward and easier to maintain.

The following example illustrates a working general section:

\begin{quote}
\begin{verbatim}
general
  version 0.5.0
  # Path where the sounds are stored
  sound-path /home/mgilfix/peep/peep-0.5.x/sounds
end general
\end{verbatim}
\end{quote}

\section{Configuring Classes}

Classes are the mechanism for grouping servers and clients together. A
single server or client can belong to several classes simultaneously. Upon
startup, both clients and servers broadcast a string of all the classes they
belong to. If the broadcaster is a server and the receiver is a client, then
that client checks in with that server. If the broadcaster is a client, then
the appropriate servers indicate to the client that they exist.

To define a class in the configuration file, the following syntax is used:

\begin{quote}
\begin{verbatim}
class <CLASSNAME>
  port <PORT NO>      # The ports used by the clients in this class
  server <SERVER_NAME>:<PORT>
end class <CLASSNAME> # (The class <CLASSNAME> is optional)
\end{verbatim}
\end{quote}

Multiple ports and multiple servers can be specified on a single
line. Alternatively, you can have a separate \code{port} and \code{server}
entry for each client port and server. The following is an example entry
that has a single server \code{darktower} in \code{test} and \code{text}
classes:

\begin{quote}
\begin{verbatim}
class test
  port 2000                   # Broadcast to clients use port 2000
  server darktower:2001       # This says that server darktower is
                              # responsible for this class and runs
                              # on port 2001
end class test

class text
  port 2000 1999                             # Broadcast to two client ports
  server darktower:2001 snowy:2001           # Server darktower and snowy are part
                                             # of this class
end class text
\end{verbatim}
\end{quote}

Clients are added into class in their respective sections. For example,
\code{logparser} can be part of the text and test class with the following
configuration example:

\begin{quote}
\begin{verbatim}
client logparser
  class text test             # All logparser clients send data to servers
    .                         # in classes text and test
    .
    .
end client logparser
\end{verbatim}
\end{quote}

If you do not wish to use auto-discovery, then defining classes is not
necessary. One thing to note is that the hostname supplied in the class
must match either the short hostname, the fully qualified hostname, or the
ip address of the machine. If you see a warning saying you that the server
couldn't assemble an identifier string, it may be due to the name you are
supplying in the class. You can tell what \code{peepd} thinks your name is
by looking at the line similar to (Note the underlined part):

\begin{quote}
\underline{darktower} | Using INET Address: 10.0.0.2:2001
\end{quote}

\section{Configuring Events and States}

As of 0.5.0, event and state sounds are stored within directories of the
sound repository.  An event or state entry tells Peep to associate the sound
files found within the specified directory accordingly, as well as provides
some other information that allows Peep to indentify the sound, and in the
case of state sounds, how when to play the sound.

An event entry is the simplest of the two and follows a straight-forward
format:

\begin{quote}
\begin{verbatim}
event
    name <EVENT NAME>    # Name to use when referring to the event
    path <RELATIVE PATH> # The directory within the repository, relative
                         # to sound-path in the general section
end event
\end{verbatim}
\end{quote}

Such an entry tells \code{peepd} to load a new event and indentify it with
\code{name}. All sounds within the directory specified with \code{path}
are loaded and associated with the given name. When an event arrives at
the server with the event's identifier, one of the sounds is chosen at
random and played. This allows for greater variation; multiple samples from
the same bird, for instance, can be associated with a single name and the
variation allows for a more natural sounding environment.

Thus an example event

\begin{quote}
\begin{verbatim}
event
    name login
    path wetlands/events/sigh
end event
\end{verbatim}
\end{quote}

loads all the sounds within the directory
\code{$sound\_path$/wetlands/events/sigh} and associates them with the given
name. One of these sounds is then played at random when a \code{login} event
arrives at the server.

State sound entries are a little more complicated. As of 0.5.0, state
sounds use thresholds to indicate which sounds should be played at what
percentage of the measurement. While scaling the volume of state sounds was
a good indiciator of increasing activity, it didn't achieve the full effect
desired. Now with thresholds, \code{peepd} can switch between playing a loud
running stream to a pounding rain storm at the desired volume.  This makes
the effect both more dramatic and more effective.

A state sound entry contains a name identifier, following by as many
thresholds as desired. The sum of the thresholds must end at \emph{1.00}
(100\%). Threshold levels are also cumulative, meaning each entry uses the
previous entry as its lower bound, with \emph{0} implied for the first
threshold entry.

\begin{quote}
\begin{verbatim}
state
    name <STATE NAME>  # Name to use when referring to the state
    threshold
        level <PERCENTAGE B/T 0-1.00> # Upper bound of threshold
        path  <RELATIVE PATH>         # Directory within repository
        fade  <FLOATING SECONDS>      # No secs to perform linear fading
    end threshold
    threshold
    .
    .
end state
\end{verbatim}
\end{quote}

Each directory containing the state sound can contain multiple sounds
segments. \code{peepd} mixes those segments into a random, continuous output
stream. Linear fading between the sounds eases the transitions between sound
segments. This scheme provides a good means of sound compression; several
smaller segments can easily be used to produce a random-sounding continuous
output stream without eating up too much disk space.

An example state entry is as follows:

\begin{quote}
\begin{verbatim}
state
  name general-feel
  threshold
    level 0.5                      # Play for measurments between 0 - 0.5
                                   # of the maximum
    path wetlands/states/water-02
    fade 0.25
  end threshold
  threshold
    level 1.00                      # Play between 0.5-1.00 of maximum
    path wetlands/states/water-01   # This sound is more dramatic than the
                                    # first one
    fade 0.1
  end threshold
end state
\end{verbatim}
\end{quote}

The event and state identifier names can then be used in other areas of
configuration, such as the individual client configurations. The server
resolves these names and plays the appropriate sound dynamically.

The number of state sounds that the server can load are fixed though.
These are dependent upon the number of voices that are used. Currently,
with 16 voices (the server default), \code{peepd} allocates 4 voices for
state sounds. This number increases as the number of sounds are increased
accordingly.

\section{Configuring Clients}

Each client has its own configuration section in \code{peep.conf}. The
section is denoted by the name of the client and defines which classes the
client belongs to, as well as what port the clients run on. Additionally,
each client section contains a configuration section specific to the
client. Client entries have the following syntax:

\begin{quote}
\begin{verbatim}
client <CLIENT_NAME>
    class <CLASS NAME> ...  # Classes for which this client belongs
    port  <PORT NO>         # Port for client to use
    config
        ....                # Client-specific config stuff
    end config

end client <CLIENT_NAME>
\end{verbatim}
\end{quote}

The classes and port specified in the configuration section apply
to auto-discovery and only have any bearing when this has been
enabled. Multiple classes can be specifed for a single client and the client
will relay its data to all servers that are members of its classes. The
\code{config} section contains configuration information specific to
the client type, in which case, it's best to consult the documentation
associated with that specific client. For the instructional purposes, a full
example configuration for the \code{logparser} client is provided:

\tiny
\begin{quote}
\begin{verbatim}
client logparser
  class main
  port 2000
  config
    default
      groups  default,firewall
      logfile /var/log/messages
    end default

    events
      # Name      Group     Location     Priority   Notification   Pattern                                             Hosts
      http        www       255          0          info          "(GET|POST).*HTTP"                                   localhost
      code-red    www       128          0          warn          "default\.ida\?XXXX"                                 localhost
      bad-query   www       128         255         crit          "ORA-\d{5}"                                          localhost
      su-login    default   128         255         warn          "PAM_unix\[\d+\]: \(system-auth\) .* opened .* root" localhost
      su-logout   default   128         255         warn          "PAM_unix\[\d+\]: \(system-auth\) .* closed .* root" localhost
      # note that if one of the previous 2 regexes match, the following 2 will be ignored
      login       default   128         255         info          "PAM_unix\[\d+\]: \(system-auth\) .* opened"         localhost
      logout      default   128          0          info          "PAM_unix\[\d+\]: \(system-auth\) .* closed"         localhost
      ssh-logins  default   128          2          info          "sshd.*Accepted password"                            localhost
      bad-login   default   128         255         warn          "pam_unix.*: authentication failure"            localhost
      ip-deny     firewall  128         255         crit          "DENY"                                               localhost
    end events
  end config
end client logparser
\end{verbatim}
\end{quote}
\normalsize

The names used in the events section of this example refer to other event
entries with the \code{peep.conf} file.

The client library documentation will provide much more thoroughly detailed
examples of how to create client configuration sections in \code{peep.conf},
so looking there is a must.

\section{Importing from within \code{peep.conf}}

Because the Peep configuration file can get rather large in size, it is
desirable to break it up into sections. The \code{import} directive allows
this to happen. Simply, chunks of the conifguration file can be put into a
separate file and then an import statement tells the servers and clients
to continue parsing using that file and then continue through the rest of
\code{peep.conf} when done.

The example client configuration for \code{logparser} in the previous
section is best put into it's own file for clarity. If that file is called
\code{peep.logtab}, we can import it with the following import statement in
\code{peep.conf}:

\begin{quote}
\begin{verbatim}
import peep.logtab
\end{verbatim}
\end{quote}

or

\begin{quote}
\begin{verbatim}
import /home/mgilfix/peep/peep-0.5.x/conf/peep.logtab
\end{verbatim}
\end{quote}

Imports are done relative to the paths of the parsed configuration files,
so a full path is usually not needed.

\section{Importing themes}

Themes are special XML files introduced as of 0.5.0 that describe a
configuration environment portably. They are described more thoroughly
in their respective section in this document. However, for now, it is
suffice to say that themes can be loaded via a similar directive to the
\code{import} statement:

\begin{quote}
\begin{verbatim}
theme themes/wetlands.xml
\end{verbatim}
\end{quote}

or

\begin{quote}
\begin{verbatim}
theme /home/mgilfix/peep/peep-0.5.x/themes/wetlands.xml
\end{verbatim}
\end{quote}

\section{An example configuration}

Other sections have described the configuration file in bits and pieces. This section
features a fully working configuration file.

\tiny
\begin{quote}
\begin{verbatim}
#
# Peep (the network auralizer) - Main configuration file
#

general
  version 0.5.0
  # Path where the sounds are stored
  sound-path /home/olsonco/peep/sounds
end general

class main
  port 2000
  server darktower:2001
end class main

# Client configurations to import
import peep.logtab
import peep.sysmontab

# Note that the pinger may very well run on the same box as the
# server (at least the way I do it).
client pinger
  class home
  port 1999
  default
    groups default
  end default
  config

    # Specify a set of hosts who will be pinged.  The event will be
    # triggered if the host is unresponsive.

    # See above for an explanation of the Event, Group, Notification,
    # or Hosts fields.

    hosts
       # Hostname or IP    Event        Group    Location   Priority   Notification  Hosts
       128.138.173.15      ping-cournot default  255        0          crit          localhost
       www.yahoo.com       ping-yahoo   default  255        0          crit          localhost
    end hosts
  end config
  # See above for more information on notification settings
  notification
    notification-hosts localhost
    notification-recipients collin.starkweather@collinstarkweather.com
    notification-level warn
  end notification
end client pinger

# Load the wetlands theme
theme themes/wetlands.xml

# Additionally load these two events
event
  name annoying
  path wetlands/events/sigh
end event

event
  name http
  path wetlands/events/sigh
end event

event
  name bad-login
  path wetlands/events/redwing-blackbird
end event

event
  name another
  path wetlands/events/redwing-blackbird
end event

event
  name ping-cournot
  path wetlands/events/redwing-blackbird
end event

event
  name ping-yahoo
  path wetlands/events/sigh
end event

# Additionally load this state
state
  name general-feel
  threshold
    level 0.5
    path wetlands/states/water-02
    fade 0.25
  end threshold
  threshold
    level 1.00
    path wetlands/states/water-01
    fade 0.1
  end threshold
end state
\end{verbatim}
\end{quote}
\normalsize

\chapter{Peep Themes}

This chapter describes what Peep theme files are, how to create them, and
how to use them.

\section{What Are Themes?}

Peep themes are the preferred format for capturing a Peep configuration
in a portable manner. They provide an easy means for exchanging a Peep
configuration from one system to another. They also provide a formal
way of creating a description of a system.

The aim of themes is to make configuration files portable across Peep
installations. An example scenario arose when the nimbda virus had just come
out. Collin Starkweather had tailored his configuration file to allow
himself to monitor the effects of the virus on his webserver. Unfortunately,
there was no easy way to share that configuration file. With the advent of
theme files, Collin could now create a nimda theme file that can easily be
added to the configuration of an existing Peep installation, allowing a
system administrator to quickly tune his system to monitor the same things
as Collin with no effort.

Themes also aim to capture more than just the necessary Peep configuration
data needed to implement the same monitoring scheme. Rather, they also
encode the thought process behind creating the sound ambiance, and the
role each sound plays.

Finally, themes can be used to catalogue sounds and give them labels.  In
the future, a sound browser could take advantage of these descriptions to
make sound selection much easier.

\section{Theme File Syntax}

Theme files are encoded in XML for ease of parsing
and maximum interoperability. They follow the XML
encoding guidelines set out by the W3C standard at:
\htmladdnormallink{http://www.w3.org/XML}{http://www.w3.org/XML}.  If that
sounds scary, don't worry, because after looking at a few examples, it's
easy to get into the groove.

An XML theme file uses the following layout and general sections:

\begin{quote}
\begin{verbatim}
<?xml version='1.0' encoding='UTF-8' standalone='no'?>

<theme name="THEME NAME">
  <sounds>
    <!-- Sound entries fo here -->
    <sound>
      <name>SOUND NAME</name>
      <category>CATEGORY DESCRIPTION</category>
      <type>TYPE OF SOUND DESCRIPTION</type>
      <format>FORMAT OF SOUND FILE</format>
      <description>DESCRIPTION OF SOUND ROLE</description>
      <path>PATH IN REPOSITORY</path>
    </sound>
    ...
  </sounds>
  <events>
    <!-- Event entries as in peep.conf with added description -->
    <event>
      <name>EVENT NAME</name>
      <sound>SOUND INDENTIFIER</sound>
      <description>EVENT DESCRIPTION</description>
    </event>
    ...
  </events>
  <states>
    <!-- State entries as in peep.conf with added description -->
    <state>
      <name>STATE NAME</name>
      <description>STATE DESCRIPTION</description>
      <threshold>
        <level>LEVEL PERCENTAGE</level>
        <fade>LINEAR FADING SECONDS</fade>
        <sound>SOUND INDENTIFIER</sound>
      </threshold>
      ...
    </state>
    ...
  </states>
  <clients>
    <!-- Description of client configurations goes here.
         Same format as peep.conf essentially -->
    <client name="CLIENT NAME">
       <!-- Client specific stuff goes here -->
       ...
    </client>
    ...
  </clients>
</theme>
\end{verbatim}
\end{quote}

All theme files begin with a standard XML header of

\begin{quote}
\begin{verbatim}
<?xml version='1.0' encoding='UTF-8' standalone='no'?>
\end{verbatim}
\end{quote}

Comments are enclosed in the usual XML style of \code{$<!-- \;\;\; -->$}.
Not all sections are required to create a theme file. If the aim of the
theme file is to categorize sound files, then only the \code{sounds}
section would be required. To share a portable server configuration, the
\code{sounds} section and \code{events} and \code{states} sections are
required. The \code{clients} section can also be included for a more
specific description.

The function of each section closely models the format used in
\code{peep.conf}.  Thus, transforming an existing configuration file to
a Peep theme should be relatively straight-forward. The only noticeable
different is the \code{name} attribute that is used in both the \code{theme}
and \code{client} tag. These provide additional descriptors that qualify the
tag and are used for indentification purposes for the servers and clients
parsing the theme. For example, the tag:

\begin{quote}
\begin{verbatim}
<client name="logparser">
  ...
</client>
\end{verbatim}
\end{quote}

creates a client entry for the \code{logparser} client. When parsing the
theme file, \code{logparser} will look for client entries that have its
name set in the \code{name} attribute of the \code{client} tag. For more
information on creating theme files and embedding client configuration,
please refer to the client documentation (\code{perldoc Net::Peep::Data}).

\section{Including Themes in Your Configuration}

Including themes in an existing Peep configuration is easy. One simply
needs to add the \code{theme} directive to an existing configuration
file. While this is discussed in slightly more detail in a previous
section, a simple example is provided here.

\begin{quote}
\begin{verbatim}
theme /home/mgilfix/peep/peep-0.5.x/themes/collin.xml
\end{verbatim}
\end{quote}

\section{A Fully Functional Example}

This section provides a fully functional theme file provided by
Collin Starkweather to demonstrate the power of themes and
make testing a new Peep installation easy.

\tiny
\begin{quote}
\begin{verbatim}
<?xml version='1.0' encoding='UTF-8' standalone='no'?>

<theme name="collin">
  <sounds>
    <sound>
      <name>whistle-01</name>
      <category>wetlands</category>
      <type>coupled</type>
      <format>raw</format>
      <description></description>
      <path>wetlands/coupled/whistle-01</path>
    </sound>
    <sound>
      <name>light-chirps-04</name>
      <category>wetlands</category>
      <type>events</type>
      <format>raw</format>
      <description></description>
      <path>wetlands/events/light-chirps</path>
    </sound>
    <sound>
      <name>thrush-01</name>
      <category>wetlands</category>
      <type>events</type>
      <format>raw</format>
      <description></description>
      <path>wetlands/events/thrush</path>
    </sound>
    <sound>
      <name>doorbell</name>
      <category>misc</category>
      <type>events</type>
      <format>raw</format>
      <description>A "ding dong" doorbell sound</description>
      <path>misc/events/doorbell</path>
    </sound>
    <sound>
      <name>rooster</name>
      <category>misc</category>
      <type>events</type>
      <format>raw</format>
      <description>A rooster crow</description>
      <path>misc/events/rooster</path>
    </sound>
    <sound>
      <name>water-01</name>
      <category>wetlands</category>
      <type>states</type>
      <format>raw</format>
      <description>The sound of trickling water</description>
      <path>wetlands/states/water-01</path>
    </sound>
    <sound>
      <name>water-02</name>
      <category>wetlands</category>
      <type>states</type>
      <format>raw</format>
      <description>The sound of rushing water</description>
      <path>wetlands/states/water-02</path>
    </sound>
  </sounds>
  <events>
    <event>
      <name>http</name>
      <sound>whistle-01</sound>
      <description>Anybody who accesses a web page will trigger this event</description>
    </event>
    <event>
      <name>code-red</name>
      <sound>thrush-01</sound>
      <description>The Code Red exploit will trigger this event</description>
    </event>
    <event>
      <name>su-login</name>
      <sound>light-chirps-04</sound>
      <description>If someone logs in as root, this event will be triggered</description>
    </event>
    <event>
      <name>su-logout</name>
      <sound>thrush-01</sound>
      <description>When someone logs out of a root account, this event will be triggered</description>
    </event>
    <event>
      <name>login</name>
      <sound>doorbell</sound>
      <description>When someone logs in, this event will be triggered</description>
    </event>
    <event>
      <name>logout</name>
      <sound>rooster</sound>
      <description>When someone logs out, this event will be triggered</description>
    </event>
    <event>
      <name>ping-yahoo</name>
      <sound>doorbell</sound>
      <description>When www.yahoo.com can't be pinged, a doorbell will ring</description>
    </event>
    <event>
      <name>red-hat</name>
      <sound>rooster</sound>
      <description>When Red Hat (RHAT) goes above or below a certain price, a rooster will crow</description>
    </event>
  </events>
  <states>
    <state>
      <name>loadavg</name>
      <description>
      The sound of running water gets progressively stronger 
      as the 5 minute load average increases
      </description>
      <threshold>
        <level>0.5</level>
        <fade>0.5</fade>
        <sound>water-01</sound>
      </threshold>
      <threshold>
        <level>1.0</level>
        <fade>0.25</fade>
        <sound>water-02</sound>
      </threshold>
    </state>
  </states>
  <clients>
    <client name="logparser">
      <options>
        <option>
          <name>port</name>
          <value>1999</value>
        </option>
        <option>
          <name>logfile</name>
          <value>/var/log/messages,/opt/apache/logs/access_log</value>
        </option>
        <option>
          <name>groups</name>
          <value>default,www</value>
        </option>
      </options>
      <configuration><![CDATA[
        # All patterns matched are done using Perl/awk matching syntax
        #  Commented lines are ones that BEGIN with a '#'
        #  Reference letters are used for the default section as well as
        #  the command line options -events and -logfile.

        events
            # Name      Group     Location     Priority   Notification   Pattern                                 Hosts
            http        www       255          0          info          "(GET|POST).*HTTP"                       localhost
            code-red    www       128          0          warn          "default\.ida\?XXXX"                     localhost
            bad-query   www       128         255         crit          "ORA-\d{5}"                              localhost
            su-login    default   128         255         warn          "\(pam_unix\)\[\d+\]: .* opened .* root" localhost
            su-logout   default   128         255         warn          "\(pam_unix\)\[\d+\]: .* closed .* root" localhost
            # note that if one of the previous 2 regexes match, the following 2 will be ignored
            login       default   128         255         info          "\(pam_unix\)\[\d+\]: .* opened"         localhost
            logout      default   128          0          info          "\(pam_unix\)\[\d+\]: .* closed"         localhost
            ip-deny     firewall  128         255         crit          "DENY"                                   localhost
        end events
	]]>
        </configuration>
        <notification level="warn">
          <hosts>
            <host name="localhost" ip="127.0.0.1"/>
          </hosts>
          <recipients>
            <recipient email="bogus.user@bogusdomain.com"/>
            <recipient email="bogus.admin@bogusdomain.com"/>
          </recipients>
        </notification>
    </client>
    <client name="sysmonitor">
      <configuration>
        # You can figure out what options these correspond to by doing a Uptime -help
        sleep      60         # Number of seconds to sleep
        loadsound  loadavg    # The sound to use when playing a load sound
        userssound users      # The sound to use for number of users
        maxload    2.5        # The initial value to consider maximum load
        maxusers   200        # The initial value to consider a lot of users
        loadloc    128        # Stereo location for load sounds
        usersloc   128        # Stereo location for user sounds
      </configuration>
    </client>
  </clients>
</theme>
\end{verbatim}
\end{quote}
\normalsize

\chapter{Developer's Guide}

This section is intended for developers or potential developers that want to
become familiar with the internals of Peep (How it works, how the code is
arranged, etc.). If you're looking to get involved in the project, this is
the section to read. This section is current as of the major code rewrite
for version 0.5.0.

\section{How to Contribute}

Probably the best way to begin getting involved
with the project is to join the mailing list at
\htmladdnormallink{http://lists.sourceforge.net/lists/listinfo/peep-develop}
{http://lists.sourceforge.net/lists/listinfo/peep-develop} and begin posting
your suggestions/comments. If you have any patches that you'd like to submit
or would like to get involved developing code, please send email my way
\htmladdnormallink{mgilfix@eecs.tufts.edu}{mailto:mgilfix@eecs.tufts.edu}
or Collin Starkweather's way at
\htmladdnormallink{collin.starkweather@colorado.edu}{mailto:collin.starkweat
her@colorado.edu}.

If you'd like to report some bugs, please use the
bug tracking utility on the sourceforge site at
\htmladdnormallink{http://www.sourceforge.net/projects/peep}{http://www.sour
ceforge.net/projects/peep} That way we can fix your bugs for future
releases.

Also, if you'd like to contribute to the maintenance of documentation, have
suggestions or ideas, I recommend sending them through the mailing list or
posted them in the public forms found on the Sourceforge page.

One last thing: if you're interested in writing/providing code, I do use a
simple naming convention in the C code. That convention is as follows:

\begin{quote}
\begin{verbatim}
Variables:      All lowercase, delimeted with '_'.
                Ex: var_foo or var

Function Names: Start lowercase, Combined Words are punctuated with caps.
                Ex: hello () or multipleWords ()

structs:        Same as variables.
                Ex: struct this_is_a_struct

typedefs:       All caps for new types
                Ex: typedef { } EVENT or EVENT thing
\end{verbatim}
\end{quote}

\section{Overview of Peep's Internals}

\emph{Note: Much has changed with the advent of the 0.5.0 release. This
release is a complete rewrite of the server code. Rereading this
documentation should quickly get anyone up to speed.}

The inner workings of \code{peepd} are based upon the interactions of three
execution threads: the server, the sound engine, and the mixer. The server
part handles all communications, including auto-discovery and keeping track
of client leases when using UDP. Upon receipt of an event, the server places
the event into a queue to be processed by the sound engine. The engine
works closely in conjunction with the mixer to keep track of the priority
of incoming and currently playing sounds. The engine also tries to find
the best available mixing channel on which to play the incoming events and
informs the mixer of the necessary parameters to properly represent the
information. Should a suitable mixing channel not be found, the engine
will place the events into a priority queue, ensuring that the mixer will
play the most important events as soon as mixing channels free up. The
mixer performs the processing necessary to produce Peep's output. This
process involves scaling each sound's volume, as well as fading between
state sounds. The mixer must also check the engine's event queue and ensure
that queued, older events have priority as soon as mixing channels free up.

\begin{rawhtml}
<P>A high level illustration of Peep's internals are shown below.</P>
<IMG SRC="./peep-internals.gif">
\end{rawhtml}

Each file in the Peep server has the following functions:

\begin{itemize}

\item alsa.c
\begin{itemize}
\item Contains the code for interfacing with the ALSA libraries. This
      code works with version of ALSA that ships with most current
      Linux distros (0.9.x).
\end{itemize}

\item cmdline.c
\begin{itemize}
\item Contains the code for parsing command-line routines and populating
      an options data structure. This data structure is then referenced
      in the main program to determine which options were set and what
      their respective values are. The interface used is similar to
      GNU's command line option generator but avoids GNU for portability.
\end{itemize}

\item cmdline.h
\begin{itemize}
\item Contains the definition of the command-line argument structure
      that contains fields indicating whether options were set and their
      respective values.
\end{itemize}

\item copyright.h
\begin{itemize}
\item Contains a character string that includes the copyright disclaimer
      so the disclaimer is viewable when the \code{strings} utility is
      run on the binary.
\end{itemize}

\item debug.c
\begin{itemize}
\item Debugging routines as well as the assert macro for assertion
      debugging. The code here is fairly self-explanitory.
\end{itemize}

\item debug.h
\begin{itemize}
\item Defines 10 different debugging flags and the 6 different debugging
      levels that provide an output that combines the flags. Also provides
      definitions for the debugging routines.
\end{itemize}

\item engine.c
\begin{itemize}
\item Contains the sound engine that determines which events play where in
      the mixer datastructure. If no channel is found, the event is enqueued
      in the engine queue for later playback and time stamped. The algorithm
      is pretty well detailed in the comments.
\end{itemize}

\item engine.h
\begin{itemize}
\item Contains the definitions for engine.c but \emph{also} contains the
      definition of an \code{EVENT}, with a description of what each field is.
      The sound table for mapping event identifiers to internal reference
      numbers is also implemented within the engine.
\end{itemize}

\item engine\_queue.c
\begin{itemize}
\item Contains the queuing structure between the events received by the server,
      waiting to be processed by the engine thread.
\end{itemize}

\item engine\_queue.h
\begin{itemize}
\item Definitions for the queue.
\end{itemize}

\item main.c
\begin{itemize}
\item Starts up the server by initialize the command-line parsing routines,
      and checking options. It then daemonizes itself if needed, and starts
      the engine and mixer threads. Finally, it acts as the server thread.
      This file also contains the main shutdown routine.
\end{itemize}

\item main.h
\begin{itemize}
\item Just some definitions for stuff in main.c, as well as some global
      settings. Also contains some default paths to files, etc.
\end{itemize}

\item mixer.c
\begin{itemize}
\item Contains all of the mixer functionality, as well the sound effects
      architecture. The sound effects architecture is a flexible way to
      add new real-time effects to Peep and currently features the linear
      fading code.
\end{itemize}

\item mixer.h
\begin{itemize}
\item Definitions for the mixer and the defined effects flags.
\end{itemize}

\item mixer\_queue.c
\begin{itemize}
\item Servers as the window for holding events that could not be played
      when received. The events are put here by the engine thread and
      time stamped to be later played when a mixer channel frees up.
\end{itemize}

\item mixer\_queue.h
\begin{itemize}
\item Definitions for the mixer queue.
\end{itemize}

\item notice.c
\begin{itemize}
\item This contains the code for creating a notice data structure, which
      is parsed from an XML notice string. This module also contains the
      functionality needed to insert code hooks into \code{peepd} to
      allow for post-processing.
\end{itemize}

\item notice.h
\begin{itemize}
\item Contains the definition of a notice data structure used for
      post-processing of an event.
\end{itemize}

\item oss.c
\begin{itemize}
\item Contains code for interfacing with \code{/dev/audio} and provides an
      abstraction for writing raw PCM to the device driver. This is one of
      the sound modules provided.
\end{itemize}

\item parser.c
\begin{itemize}
\item Contains the code to parse the \code{peep.conf} style configuration
      format. It also provides some common routines to the XML theme
      parser.
\end{itemize}

\item parser.h
\begin{itemize}
\item Contains definitions for the parser code.
\end{itemize}

\item playback.c
\begin{itemize}
\item Contains all the code for recording events and performing playback.
      The playback code hooks into the engine code in an if statement that
      decides whether or not the event needs to be recorded/is ready for
      playback.
\end{itemize}

\item playback.h
\begin{itemize}
\item Definitions for the playback code, including the definition of the
      playback file header.
\end{itemize}

\item server.c
\begin{itemize}
\item Contains the server routines common to all server modules that
      implement specific server protocols. This includes initializing
      the UDP port for broadcasting.  Finally, it calls the real server
      module functions to perform the actual connection handling.
\end{itemize}

\item server.h
\begin{itemize}
\item Contains the \emph{Peep protocol definitions}. These include the
      definition of a packet header, as well as packet contents.
\end{itemize}

\item sound.h
\begin{itemize}
\item General sound header that provides an the definitions of the
      abstractions for the sound modules.
\end{itemize}

\item ssl\_server.c
\begin{itemize}
\item Implements the server routines that use SSL as the main communication
      protocol.
\end{itemize}

\item ssl\_server.h
\begin{itemize}
\item Contains definitions for the SSL server routines.
\end{itemize}

\item tcp\_server.c
\begin{itemize}
\item Implements the server routines that use TCP as the main communication
      protocol.
\end{itemize}

\item tcp\_server.h
\begin{itemize}
\item Contains definitions for the TCP server routines.
\end{itemize}

\item thread.c
\begin{itemize}
\item Wraps POSIX threading code into simple functions. Also wraps semaphore
      code.
\end{itemize}

\item thread.h
\begin{itemize}
\item Contains the definiton for the threading functions.
\end{itemize}

\item udp\_server.c
\begin{itemize}
\item Implements the server routines that use UDP only as the main
      communication protocol, as well as the leasing needed for client
      management.
\end{itemize}

\item udp\_server.h
\begin{itemize}
\item Contains definitions for the UDP only server routines as well as the
      leasing protocol definitions.
\end{itemize}

\item xml.c
\begin{itemize}
\item Contains some handy routines to clean up strings for XML parsing.
\end{itemize}

\item xml.h
\begin{itemize}
\item Definitions of the XML helper routines.
\end{itemize}

\item xml\_notice.c
\begin{itemize}
\item Contains the code for parsing XML notice messages received
      from client events and populates the notice data structure,
      which is then passed to the server code hooks.
\end{itemize}

\item xml\_notice.h
\begin{itemize}
\item Contains the defintion of the XML tags, as well as the parsing routines.
\end{itemize}

\item xml\_theme.c
\begin{itemize}
\item Contains the code for parsing XML theme files and loading the appropriate
      sounds. It uses some of the convenience routines available in parser.c.
\end{itemize}

\item xml\_theme.h
\begin{itemize}
\item Contains the defninition of the XML tags, as well as the theme parsing
      routines.
\end{itemize}

\end{itemize}

The next few sections aim to provide a general idea of how the code is
structured and make an introduction into the project easier. These sections
will probably need to be refined over time, with detail added, so if you
have any suggestions or pieces of crucial information you felt were missing,
please send suggestions!

\section{The Server Data Structures}

The server data structures consist of \code{struct}'s that define the
contents of packets, as well as protocol constants. For the UDP server
routines, some extra data structures to holding leasing information
are also defined.

The data structures common to all server types are:

\begin{quote}
\begin{verbatim}
/* Peep protocol packet */
typedef struct {
    unsigned char version; /* major protocol version */
    unsigned char type;    /* type of packet */
    unsigned char content; /* contents format */
    char reserved[5];      /* reserved */
    int magic;             /* magic number for checking */
    int len;               /* length of the contents */
} HEADER;

/* Holder structure for the packet */
typedef struct {
    HEADER header;         /* packet header */
    void *body;            /* body of the actual message */
} PACKET;

/* PACKET body types */
typedef char * XML_BUFFER;
typedef EVENT EVENT_BODY;
typedef char * MSG_STRING;

typedef struct broadcast {
    int port;
    struct broadcast *next;
} BROADCAST;
\end{verbatim}
\end{quote}

When using UDP only, some extra data structures are defined in
\code{udp\_server.h}:

\begin{quote}
\begin{verbatim}
/* Peep leasing is a very low resolution leasing and is not meant to be
 * used for long periods of time. After lease time has expired, the
 * information associated with the broadcast should be considered
 * obsolete
 */
struct lease {
    unsigned char min;
    unsigned char sec;
};

/* The entry in the linked list database of active clients */
struct leaselist {
    struct in_addr host;       /* The ip address of the active host */
    unsigned int port;         /* The port to address the host with */
    struct timeval expired;    /* The time remaining at which this
                                * has expired
                                */
    struct leaselist *nextent; /* The next entry in the list */
};

/* Definition for body of a packet whose sole purpose is to send
 * leasing information
 */
typedef struct {
    struct lease lease;
} LEASE_BODY;

/* Definition of a packet body for sending both lease information
 * and ID information
 */
typedef struct {
    struct lease lease;
    int len;
    char *id;
} UDP_BROADCAST_BODY;
\end{verbatim}
\end{quote}

All Peep packets contain a header structure as defined in the \code{HEADER}
struct and a variable body type. The version field of the header is used to
differntiate versions of the protocol and will mostly likely change with
future versions. The type field indicates the purpose of the packet.  In the
Peep protocol, packets can have the following purposes, as defined by
protocol constants in \code{server.h}:

\begin{itemize}
\item \code{\#define PROT\_UNDEFINED 0}
\begin{itemize}
\item The packet contents were not set. This corresponds to an invalid packet
      type.
\end{itemize}
\item \code{\#define PROT\_BC\_SERVER (1 $<<$ 0)}
\begin{itemize}
\item This indicates that the packet contains information regarding a server
      broadcast.
\end{itemize}
\item \code{\#define PROT\_BC\_CLIENT (1 $<<$ 1)}
\begin{itemize}
\item This indicates that the packet contains information regarding a client
      broadcast.
\end{itemize}
\item \code{\#define PROT\_CLIENT\_EVENT (1 $<<$ 2)}
\begin{itemize}
\item The packet contains information about a monitored event.
\end{itemize}
\end{itemize}

In addition, the UDP only communication scheme defines some extra protocol
constants in \code{udp\_server.h}:

\begin{itemize}
\item \code{\#define PROT\_SERVER\_STILL\_ALIVE (1 $<<$ 3)}
\begin{itemize}
\item Indicates that the server is still alive and sends along updated
      leasing information.
\end{itemize}
\item \code{\#define PROT\_CLIENT\_STILL\_ALIVE (1 $<<$ 4)}
\begin{itemize}
\item Indicates that the client is still alive and that new lease information
      has been sent along for updating.
\end{itemize}
\end{itemize}

The content field of the packet indiciates how the packet content has been
laid out. Different packet contents are appropriate for different packet
types. There are three defined packet types as evidenced by the \code{typdef}'s
in the header. They are:

\begin{itemize}
\item \code{\#define PROT\_CONTENT\_UNDEFINED 0}
\begin{itemize}
\item The packet content type was not set and the packet is invalid.
\end{itemize}
\item \code{\#define PROT\_CONTENT\_XML (1 $<<$ 0)}
\begin{itemize}
\item The contents of the packet are in XML. XML is used for sending client
      events and providing a rich amount of data regarding the original
      monitored source that triggered the event. Because XML packets can get
      rather bulky, a barebones packet type was defined\ldots
\end{itemize}
\item \code{\#define PROT\_CONTENT\_EVENT (1 $<<$ 1)}
\begin{itemize}
\item The packet contains an \code{EVENT} structure, as defined in
      \code{engine.h}. This is the raw event that the server uses for
      creating its audio representation of an event. This can only be used
      to signal client events.
\end{itemize}
\item \code{\#define PROT\_CONTENT\_MSG (1 $<<$ 2)}
\begin{itemize}
\item This indicates that the contents are merely a string of data. This
      format is used for Peep broadcasts to contain the class indentifier
      string.
\end{itemize}
\end{itemize}

Again, the UDP only communication scheme defines some extra packet types for
containing lease information in \code{udp\_server.h}:

\begin{itemize}
\item \code{\#define PROT\_CONTENT\_BC\_LEASE (1 $<<$ 3)}
\begin{itemize}
\item The packet contents contain both broadcast information and leasing
      information as defined in the \code{lease} struct.
\end{itemize}
\item \code{\#define PROT\_CONTENT\_LEASE (1 $<<$ 4)}
\begin{itemize}
\item The contents of the packet contain pure lease information and are used
      for updating the server's lease cache.
\end{itemize}
\end{itemize}

When staring up, the server constructs a \code{BROADCAST} packet and sets
the packet type and packet content type accordingly. It constructs a string
containing a list of its peep classes from configuration information
found in \code{peep.conf} and delimites them with a ''!''. If leasing is
being used, it also constructs a lease time which is based on the leasing
constants set in \code{udp\_server.h}. There are two lease constants: one
set tells the client when to consider the server non-functional, and the
other when the server should renew its lease with the clients.

One the clients and servers have discovered each other, further messaging
takes place via direct communication over the protocol of choice. With TCP
and SSL, a session is established and a separate server thread is spawned
to handle the new session. With UDP, leasing information is sent via direct
communication with servers on top of event information.

In the UDP only scheme, because the servers and clients broadcast only
once upon startup, the server maintains a list of clients using the
\code{leaselist} struct. After the server starts up and broadcasts its
existence and the appropriate clients respond that they're alive, the server
adds the client into the lease list. Similarly, when the clients start up,
theya re given lease information and added into the lease list. The server
also records the time when the client entry will expire. If the server
doesn't hear from the client after the lease time has elapsed, it removes
the entry from the hostlist. An alarm is scheduled at an even interval
to scan the lease list and purge any expired entries.

Finally, when a client event is received, the server performs any processing
needed on the packet, and places an \code{EVENT} defined in \code{engine.h}
into the engine queue. In the case of an XML event, the XML data is parsed
and fed into the server code hook for post processing.

\section{Creating Code Hooks}

Because Peep provides a solid platform for data gathering and will process a
considerable amount of data while monitoring, a new plug-in architecture was
introduced with the 0.5.0 release. The transition to XML messaging opened
up the pathway to easily piggy-backing additional data for real-time or
post-processing on a central server.

Upon receipt of an XML event, the server processes the event and parses
the XML into a \code{NOTICE} structure, defined in \code{notice.h}. After
the parse, the event is fed into the sound engine and then the code hook
function, \code{executeEventNoticeHook}, is executed. The function is fed
the original XML data and the pre-defined \code{NOTICE} data structure
filled from the parse.

To add a plug-in, a function call should be added into the
\code{executeEventNoticeHook} function to call the plug-in routines. Plug-in
developers need not worry if Peep doesn't understand how to parse the extra
information piggy-backed onto the initial message. \code{peepd} will attempt
to fill out the \code{NOTICE} structure as best as it can and provide the
original XML message for additional parsing.

It is important to note that \code{peepd} excutes the code hook function
inline while processing packets. That means that a function that occupies
a considerable amount of processing time will adversely affect it's
performance. To circumvent that, data can be written to a log file and
processed by another program. Or, quick operations can be performed. The
code-hook is also executed in real-time, allowing for both real-time and
post-processing. The code-hook will most likely be called by multiple
threads, so some basic thread locking might be in order. Helper routines are
supplied in \code{thread.c}.

\section{The Sound Engine}

The main data structures involved with the sound engine are found in both
\code{engine.h} and \code{engine.c}:

\begin{quote}
\begin{verbatim}
typedef struct {
    unsigned char type;      /* state or single event */
    unsigned char loc;       /* stereo location (left - 0, right - 255) */
    unsigned char prior;     /* priority of the event */
    unsigned char vol;       /* volume of the event */
    unsigned char dither;    /* adjustable parameter for sound dithering
                              * 2 meanings:
                              *   1) Applies to states. sets the fade-in time
                              *      when mixing between state sounds
                              *   2) Applies delay to handle events which
                              *      occur in spurts - currently
                              *      unimplemented
                              */
    char reserved[2];        /* reserved for future effects attributes */
    int flags;               /* effects flags */
    int sound_len;           /* Length of the sound string */
    char *sound;             /* sound to play, ref by name */
} EVENT;

typedef struct {
    EVENT event;
    struct timeval mix_time; /* time when an event was enqueued */
} ENGINE_EVENT;

struct engine_sched {
    double startt;    /* start times for each voicing */
    long priorit;     /* priority for currently playing sound */
    double minendt;   /* end time of the sound - currently unused */
};

/* Engine scheduling datastructure (calloc'd to half the number
 * of channels
 */
struct engine_sched *sched = NULL;

/* Number of event and state channels */
static unsigned int no_ebuffs = 0;
static unsigned int no_sbuffs = 0;
\end{verbatim}
\end{quote}

\code{engine.c} also contains the sound table, whose purpose is to map
names into the internal memory indices of the loaded sounds. Essentially,
this is a hash table whose records are references to the sound data
and descriptive data about the sound data.

\begin{quote}
\begin{verbatim}
typedef struct {
    short **snds;             /* array of event samples in mem */
    unsigned int *lens;       /* length of samples */
    unsigned int snd_cnt;     /* number of event sounds associated with
                               * an event
                               */
} EVENT_ENTRY;

typedef struct {
    int mixer_index;          /* The index within the internal buffer of
                               * the mixer */
} STATE_ENTRY;

struct sound_entry {
    struct sound_entry *next;  /* Pointer to the next entry in the hash
                                * list
                                */
    char *name;                /* Name of the sound to look up */
    void *data;                /* Pointer to the type of data. Depending on
                                * type, this can be a pointer to a:
                                *   EVENT_ENTRY, which contains the data
                                *   associated with an event.
                                *            or
                                *   STATE_BUF, which contains the index of
                                *   the mixer buffer that holds the states
                                */
    EVENT_TYPE type;           /* Type of the event stored */
};

/* For the sound table */
static struct sound_entry **sound_table = NULL;
\end{verbatim}
\end{quote}

The sound engine's main purpose is to assign incoming events appropriate
channels in the voice mixer. For state events, this simply means making the
appropriate changes in volume and stereo position of the voice mixer's state
data structures. For singular events, the sound engine executes an algorithm
to calculate the best available channel based on what's available, and the
priority of the sound. If no channel is available, the event is put into
a priority queue (literally based on event priority), to be played by the
voice mixer as soon as a channel frees up.

The first data structure defined in the sound engine header is the event
format. This is the format that Peep uses to represent an event internally
and is also the same format that clients can send to the server if they do
not wish to use XML representation. The event is then time stamped (stored
in the \code{mix\_time} field) to allow the sound engine to keep track of
when the sound began playing.

During initialization, the engine creates the \code{engine\_sched} data
structure, which contains entries which correspond to each event channel in
the mixer. Each entry stores the time at which the event started playing
and the event's corresponding priority. This aids the sound engine in
determining how to schedule sounds. The \code{minendt} field is currently
not used but it's original purpose was to aid in deciding when a sound could
safely be interrupted during playback and replaced with a higher priority
one. The original idea was to interrupt a sound as soon as it was considered
intelligeable, but not necessarily complete. It is possible this may be
re-introduced to the algorithm at a later date.

The sound table is used for converting the symbolic names of sounds to their
internal reference in memory. This is essentially a look up in a hash table,
which returns a records containing a pointer to the sound data and any other
information which is needed to play the sound. Two different record types
exist in the sound table: a record type for events and a record type for
states.  Because states are already loaded into the mixer and are playing
continuously, only the mixer channel used to hold the sound data needs to be
kept in the sound table. For events, the pointers to the sound data are kept
in the sound table and are fed to the mixer for playback. Other information
kept is the length of the sounds in the entry and the number of sounds
associated with the record.

The \code{mixer\_queue.c} file contains the priority queue code, which
begins filling up with events based on event priority if all channels are
taken and uninterruptable. The mixer looks at this queue directly as soon as
channels free up.


\section{The Mixer}

The mixer contains the following data structures (in \code{mixer.c}):

\begin{quote}
\begin{verbatim}
struct {
    short *snd_buf;    /* array of sound data */
    unsigned int len;  /* length of data */
    unsigned int pos;  /* current position ptr */
    double stereo_pos; /* stereo (left) */
    int filter_flag;   /* flag of effect filters to apply */
} EVENT_BUF;

/* A single state sound entry */
typedef struct {
    short **snd_buf;      /* array of state segments */
    unsigned int *len;    /* length of segments */
    unsigned int snd_cnt; /* number of sounds per state */
    unsigned int snd_no;  /* current segment to look at */
    unsigned int pos;     /* position in that segment */
} STATE_SND;

/* Threshold describing the bounds for at which an event applies */
typedef struct {
    double l_bound;       /* lower bound for the state */
    double h_bound;       /* upper bound for the state */
    STATE_SND state_snd;  /* the state sound associated with
                           * the entry
                           */
} THRESHOLD;

/* State buffer descriptor */
typedef struct {
    THRESHOLD *thresh;     /* threshold of each state segment */
    int thresh_cnt;        /* number of thresholds */
    double stereo_pos;     /* stereo (left) */
    double vol;            /* volume */
    int filter_flag;       /* flag of effect filters to apply */
} STATE_BUF;

/* Event mixing buffers */
static EVENT_BUF *ebuffs;
static unsigned int no_ebuffs = 0;

/* State mixing buffers */
static STATE_BUF *sbuffs;
static unsigned int no_sbuffs = 0

/* Array of dynamic volumes, alloc'd to no_ebuffs and zero active
 * buffer count
 */
static double *dyn_mul;
static unsigned int dyn_buf_cnt = 0;
\end{verbatim}
\end{quote}

The voice mixer doesn't just mix, but does all sorts of stuff in addition
to mixing. These things include picking the next random segment to play for
a given state sound, applying effects to sound in real-time, checking if
there are events waiting in the priority queue after a mixed event finishes
playing, and calculating event volumes dynamically. The voice mixer also
figures out how many channels get allocated to events and states. The
default ratio to use is that state mixing channels comprise of 1/4 of the
total number of voices. For 16 voices, that means that 12 events can play
simultaneously and 4 channels are reserved for state sounds. That ratio
can be changed by changing the constant \code{PERCENT\_STATE\_SOUNDS} in
\code{main.h}. \code{>peepd} defaults to 16 voices.

The mixer event data structure is an array of \code{EVENT\_BUF} records
that specify what sound is on the given event channel, what offset the
sound is currently at, the sound's stereo location, which real-time effects
to apply to the sound, and the length of the sound. Mixing is done by
calculating a given sound's stereo location, applying the appropriate
real-time effects, and then summing all the sounds together. To avoid
clipping, a dynamic mixing algroithm scales the volumes of the sounds. It
does this by keeping track of what multipliers it's currently using and the
number of voices currently playing. When a new sound comes in, it sums all
of the dynamic multipliers (the max possible is 1.0) and determines how much
headroom, or space, it can use to calculate the volume multiplier for the
new sound. The algorithm used results in the first sounds in a long sequence
sound louder than the trailing sounds, but that volumes of all sounds
quickly event out with a continuous stream of events. The \code{configure}
option \code{--with-static-volume} doesn't use the dynamic volume algorithm
and simply divides all sounds by the number of voices used to avoid any
possible clipping. This is a legacy option but could offer some performance
gains on really slow machines.

The state data structures are a little more complicated. Individual state
sounds are loaded into \code{STATE\_SND} records and are kept in the
mixer.  The mixer keeps track of which sound segment is playing, it's
stereo location, the real-time effects to apply, and the number of sound
segements associated with the event. To form a continuous stream of output,
the mixer streams the segments together, choosing the next segment at
\emph{random}. Individual state sound records are stored in a threshold
data structure. The threshold record tells the mixer which state record to
select based on the volume of the state. This allows, for example, scaling
a stream to a waterfall sound to better represent increasing activity. The
engine modifies the volume and stereo location parts of the data structure
directly and the mixer the actual volume and stereo location of the sound
are calculated dynamically while mixing.

During the mixing process, real-time effects are applied to the sound data,
depending on which flags are set in the \code{filter\_flag} member variable.
The effects architecture is outlined in detail in a subsequent section.

Some other stuff to note about the mixer is that it aims to feed enough
sound data into the audio buffers on the sound card so that it can sleep for
a bit before continuing calculations. Here, there was a tradeoff between
response time and performance. The ''chunk size'' used in this case is
roughly a 1/2 second of sound. The mixer sleeps for about half of that and
then begins calculating again. The idea is to always keep the audio buffer
on the sound card while maximizing sleep time. The current compromise works
alright but could be tweaked...

Finally, the voice mixer interfaces with actual sound devices through an
abstraction layer, which is introduced in a subsequent section.

\section{The Effects Architecture}

The way effects are applied to sound data has been completely re-designed
with the advent of the 0.5.0 version of Peep. The effects architecture
uses flags to determine which effects to apply. These flags are defined in
\code{mixer.h}. Currently, only one dynamic effect has been implemented:
linear fading. The definition of these flags in the header are as follows:

\begin{quote}
\begin{verbatim}
/* Filter flag definitions. Not that both event and state filter flags
 * should have a MAX_*_FLAG at the end of the bunch
 */

#define MAX_EVENT_FLAG (1 << 0)

#define STATE_LINEAR_FADE_FLAG (1 << 0)
#define MAX_STATE_FLAG (1 << 1)
\end{verbatim}
\end{quote}

The max flag is provided for speed so that the mixer code need not search
continuously through all possible flags in a 32-bit integer while processing
the sound code. To add a new flag, the max flag need only be increased and
the new flag inserted in it's place.

Effects are called in the \code{mixerApplyEventFilters} and
\code{mixerApplyStatFilters} functions in \code{mixer.c}. These functions,
when called, have access to the array of sound data to process, the current
possition in the data, and index of the actual sound buffer (\code{ebuff}
or \code{sbuff}) record. This function then iterates through the flags and
using a case statement, determines which effects to apply.

Each effect has full access to all the sound data during play and can follow
the current position of the mixer in the sound. This allows for a wide-range
of real-time processing and considerable flexibility in structuring effect
algorithms. The currently implemented linear fading effect is a good
example, and it suggested looking at the three functions that are part of
the implementation in

\codeblock{short mixerFadeEffect (short data, unsigned int pos, unsigned int j);}

for details.

Adding a new effect is easy. First, a new flag is defined in
\code{mixer.h}, as well as the appropriate function prototypes. A new
\code{case} statement is added to the \code{switch} statement in either
\code{mixerApplyEventFilters} or \code{mixerApplyStateFilters}. New effects
will probably want to add initialization functions and shutdown functions,
which should be added to the \code{mixerInit} and \code{mixerShutdown}
functions respectively. The effect function can then store any state
information and should always return the newly processed \code{short} that
representes the currently sound value.

\section{The playback code}

Events are recorded into a ''round-robin'' within the playback file. The
structure for the playback file is (excuse my cheesy ansi art):

\begin{verbatim}
 Header (struct playback_h in playback.h always first
            sizeof(struct playback_h) bytes of file)
            |
            |
         Event loop (Number of events set by  ----|
                   a #define in Playback.h)       |
           |                                      |
           ----------------------------------------
\end{verbatim}

The events are recorded as they are fed to the sound engine (in
\code{engine.c}), and are time stamped write before they are written
to the event log, using the \code{mix\_time} field of the \code{EVENT}
structure. One thing to note is that the structure contains a start position
field which is set and written out to the file in the header section that
specifies where the round-robin in the file begins. For robustness, if the
header isn't written out upon shutdown (for whatever reason), the playback
code can search through the file and determine where the round-robin begins
based on the \code{mix\_time}'s. When \code{peepd} is started in playback
mode, rather than starting the server thread, a playback thread starts and
reads events in the file and feeds them to the engine thread just as though
they were actually received by the server.

\section{The Sound Module Interface}

The sound modules are basically a set of function calls and constants that
abstract the actual sound device underneath. The two types of sound modules
currently implemented are \code{/dev/audio} support and ALSA support(for
0.9.x). Constants for those device drivers are abstracted via the definition
of the constants in the \code{sound.h} header:

\begin{quote}
\begin{verbatim}
AU_SOUND_FORMAT
SIGNED_16_BIT
SOUND_WRONLY
SOUND_RDONLY
SOUND_RDWR
\end{verbatim}
\end{quote}

The \code{sound.h} header also defines other data structures capabable of
describing the audio hardware. However, these structures cannot be filled
out with \code{/dev/audio} support, which is the main thrust of support at
the moment, and so are not really taken advantage of. The main functions to
implement to create a sound modules are:

\begin{quote}
\begin{verbatim}
void *soundInit (void *snd_device, int mode);

int soundSetFormat (void *handle, unsigned int format_type,
                    unsigned int rate, unsigned int chans,
                    unsigned int port);

ssize_t soundPlayChunk (void *handle, char *buf, unsigned int len);

void soundClose (void *handle);
\end{verbatim}
\end{quote}

These have pretty obvious purposes. The current mixing scheme does
everything in software. While this is portable across all hardware, it
means that a lot of processing is CPU bound, rather than offloaded to the
hardware. The two other functions defined in \code{sound.h} could be used
to figure out what sort of capabilities the hardware offers and to take
advantage of hardware directly in future versions.

\section{Making Sounds for Peep}

If you're planning on making sounds for Peep, I suggest
looking into a program called Goldwave for sound processing:
\htmladdnormallinkfoot{goldwave}{http://www.goldwave.com}. If you're
processing event sounds, which tend to be higher in pitch by nature, I
suggest running a high-pass filter at 1 Khz, to get rid of a lot of the low
end noise. Finally, please remember to normalize your sounds so that they
have maximum volume.

The sound format used by Peep is PCM raw signed-16 bit stereo at 44100 Khz
(CD-quality), which has the extension .snd. To associate sounds with a
single event or state, simply put all the sounds under a single directory in
the sound repository.

\chapter{Troubleshooting}

This section aims at providing solutions for common problems. This section
will expand as I get feedback from you and figure out what those common
problems are =) It's best to check the \htmladdnormallinkfoot{project
page}{http://www.sourceforge.net/projects/peep} though, as it will get
updated much more frequently than this document.

\section{Couldn't Assemble Identifier String}

This message means that \code{peepd} couldn't find or recognize the hostname
of the machine. \code{peepd} does look for both the machine's local name,
it's ip address, and fully qualified host name in the configuration file
while searching through classes. If for some reason this still happens to
you, I suggest doing a \code{uname -n} to see what \code{peepd} thinks your
machine is called, and then using that machine name within \code{peep.conf}.

\end{document}
